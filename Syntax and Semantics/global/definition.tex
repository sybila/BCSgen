\documentclass{entcs}
\usepackage{prentcsmacro}
\usepackage{kpfonts}
\usepackage{hyperref}
\usepackage{algorithmicx}
\usepackage{graphicx}
\usepackage[noend]{algpseudocode}

\hyphenation{me-ta-bo-lism}

% global macro definitions
\newcommand{\TBD}{\textbf{TBD}~}
\newcommand{\spec}[1]{\texttt{#1}}
\renewcommand{\~}[0]{\texttildelow}
\renewcommand{\algorithmiccomment}[1]{\hfill\textit{\# #1}}
\newcommand*\arc{{\fontfamily{pbk}\fontseries{db}\selectfont+}}
\newcommand{\pipe}{{}{\scalebox{.6}{$|$}}{}}
\newcommand{\choice}{|}

\newtheorem{notation}[thm]{Notation}

\def\lastname{Troj\'{a}k et al.}

\begin{document}

% title page
\begin{frontmatter}
\title{Formal Biochemical Space Language}

\author{\normalsize
T. D\v{e}d, D. \v{S}afr\'anek, M. Troj\'ak, M. Klement, J. \v{S}alagovi\v{c}, L. Brim}
\address{Faculty of Informatics, Masaryk University\\
Brno, Czech Republic
}

\end{frontmatter}

\section{Biochemical Space Agents}

Let $\mathcal{N}_{a},~\mathcal{N}_{T},~\mathcal{N}_{x},~\mathcal{N}_{c},~\mathcal{N}_{s}$ be mutually exclusive finite sets of atomic, structure, complex, compartment, and state names, respectively. 

\subsection{Atomic agents}

Agents are defined hierarchically starting from \emph{atomic agents}. These agents are the smallest units in our hierarchy a represent ...

\begin{definition}
Atomic agent $\mathtt{a}$ is defined as triple ($\alpha, \delta, \mathtt{c}$) where $\alpha \in \mathcal{N}_{a}$ is name, $\delta \subseteq \mathcal{N}_{s}$ is finite non-empty set of states $\mathtt{s} \in \mathcal{N}_{s}$ in which the agent can occur, and $\mathtt{c} \in \mathcal{N}_{c}$ is physical compartment within which it is considered.
\end{definition}

Atomic agent expressions have the following syntax:

\begin{center}
{\small
\hspace*{-1cm}\begin{tabular}{ ll ll ll ll }
 atomic agent & $\mathtt{a} ::= \alpha\{\mathtt{s}\}::\mathtt{c}~\choice~\alpha::\mathtt{c}$ & state & $\mathtt{s} ::= n \in \delta$\\
 name & $\alpha ::= n \in \mathcal{N}_{a}$ & compartment & $\mathtt{c} ::= n \in \mathcal{N}_{c}$\\
\end{tabular}
}
\end{center}

\begin{definition}
Let $\mathtt{a},~\mathtt{a}'$ be atomic agents. We define the \emph{structural equivalence of atomic agents} by claiming $\mathtt{a}\equiv\mathtt{a}'$ whenever $\mathtt{\alpha} = \mathtt{\alpha}'$, $\mathtt{c} = \mathtt{c}'$ and $\delta = \delta'$.
\end{definition}

\begin{notation}
{~}
\begin{itemize}
\item We denote $\mathtt{s}\in\delta$ the fact that $\mathtt{s}$ is included in the state signature~$\delta$.
\item Syntax $\alpha::\mathtt{c}$ is used for expressing atomic agent when state is unknown, i.e. all states $\mathtt{s} \in \delta$ are considered.
\end{itemize}
\end{notation}

\begin{defn}
Let $\mathtt{a},\mathtt{a}'$ be atomic agents. We say agent $\mathtt{a}=(\alpha, \delta, \mathtt{c})$ is \emph{compatible with} agent $\mathtt{a}'=(\alpha', \delta', \mathtt{c}')$, written $\mathtt{a} \lhd \mathtt{a}'$, iff $\alpha = \alpha'$, $\mathtt{c} = \mathtt{c}'$, and $\delta \subseteq \delta'$. 
\end{defn}

\begin{theorem}
Since atomic agents are constructed from finite sets $\mathcal{N}_{a},~\mathcal{N}_{c}$ and $\mathcal{N}_{s}$, we can define universe of atomic agents $\mathcal{A} = \{ \mathtt{a}~|~\mathtt{a} = (\alpha, \delta, \mathtt{c}) \}$.
\end{theorem}

\subsection{Structure agents}

Next we proceed with defining \emph{structure agents}. A structure agent represents a biochemical object that is composed from several known atomic agents provided that we know that such a composition is abstract and not necessarily complete. To incorporate such an abstraction of biological structures into our language, a structure agent is defined to be labelled with a unique name and it is constructed only from atomic agents considered in the same physical compartment. 

A typical example of a structure agent is a protein where the atomic agents are individual amino acids that are of interest in the particular setting.

\begin{definition}
Structure agent $\mathtt{T}$ is defined as triple ($\tau, \gamma_p, \mathtt{c}$) where $\tau \in \mathcal{N}_{T}$ is name, $\gamma_p \subseteq \mathcal{A}$ is finite set of atomic agents $\mathtt{a} \in \mathcal{A}$ called \emph{partial composition}, and $\mathtt{c} \in \mathcal{N}_{c}$ is physical compartment within which it is considered.
\end{definition}

Structure agent expressions have the following syntax:

\begin{center}
{\small
\hspace*{-1cm}\begin{tabular}{ ll ll }
 structure agent & $\mathtt{T} ::= \tau(\gamma_p)::\mathtt{c}~\choice~\tau::\mathtt{c}$\\
 structure name & $\tau ::= n \in \mathcal{N}_{T}$\\
 partial composition & $\gamma_p ::= \mathtt{a},~\gamma_p~\choice~\mathtt{a}$\\
\end{tabular}
}
\end{center}

Note that a compartment of a structure agent is uniquely given by the compartment specified in its parts. We restrict ourselves to structure agents where all atomic agents in the partial composition have the same compartment. Assuming this restriction, we can shorten the notation by omitting compartments in the atomic agents of a partial composition.    

\begin{notation}
~
\begin{itemize}
\item We denote $\mathtt{a}\in\gamma_p$ the fact that $\gamma_p$ includes the atomic agent $\mathtt{a}$.
\item Syntax $\tau::\mathtt{c}$ is used for better readability of structure agents when $\gamma_p$ contains all defined states for the agent.
\end{itemize}
\end{notation}

\begin{defn}
Let $\mathtt{T},\mathtt{T}'$ be structure agents. We define the \emph{structural equivalence of structure agents} by claiming $\mathtt{T}\equiv\mathtt{T}'$ iff there exist $\tau,\tau',\gamma_p,\gamma'_p,\mathtt{c}$ such that $\mathtt{T}=\tau(\gamma_p)::\mathtt{c},\mathtt{T}'=\tau'(\gamma'_p)::\mathtt{c}$, $\tau=\tau'$ and $\gamma_p=\gamma'_p$.
\end{defn}

%As a representative of a class of structurally equivalent structure agents we consider the agent $\mathtt{\tau}(\gamma_p)$ where the agents in $\gamma_p$ are lexicographically ordered by names. Since atomic agents cannot be repeated in a structure agent, such an order is total.

\begin{defn}
Let $\mathtt{T},\mathtt{T}'$ be structure agents. We say $\mathtt{T}$ \emph{is compatible with} $\mathtt{T}'$, written $\mathtt{T} \lhd \mathtt{T}'$, iff there exist $\tau,\tau',\gamma_p,\gamma'_p,\mathtt{c}$ such that $\mathtt{T}=\tau(\gamma_p)::\mathtt{c},\mathtt{T}'=\tau'(\gamma'_p)::\mathtt{c}$, $\tau = \tau'$ and for each atomic agent $\mathtt{a} \in \gamma_p$ there exists an atomic agent $\mathtt{a}' \in \gamma'_p:$ $\mathtt{a}~\lhd~\mathtt{a}'$. 
\end{defn}

In the following we define the last step in the hierarchy of agents. In particular, we define \textit{complex agents}. A complex agent represents a non-trivial composite biochemical object that is (inductively) constructed from already known biological objects. In common rule-based languages this is typically defined by introducing some kind of bonds between individual biochemical objects. In BCS we abstract from detailed specification of bonds and we rather assume a complex as a coexistence of certain objects in a particular group. Such a group can be optionally referred to by a unique name. A complex agent is constructed from structure agents where all are required to reside in the same compartment~$\mathtt{c}$. 



The key element of a complex agent is \emph{full composition} describing inductively constructed coexistence expressions from existing agents. We restrict ourselves to full compositions where all agents reside in the same compartment.

\begin{center}
{\small
\hspace*{-1cm}\begin{tabular}{ ll ll }
 complex agent & $\mathtt{X}::=\gamma_f::\mathtt{c}~\choice~n\in{N}_{x}::\mathtt{c}$\\
 full composition & $\gamma_f ::= \mathtt{T}.\mathtt{T}~\choice~\mathtt{T}.\gamma_f$\\
\end{tabular}
}
\end{center}

In contrast to partial compositions, we allow replication at the level of full compositions (an agent of a certain name can appear more than once in a full composition). Moreover, names of complex agents are not associated with particular full compositions at the level of agent expressions. This is done at the level of agent signatures (see Section~\ref{sec:sigs}).

Note that in similar way as in the case of structure agents, we restrict the formalism to complex agents where the compartment is the same for all agents inside the respective full composition.

\begin{notation}
~
\begin{itemize}
\item Let $\mathtt{X}=\gamma_f::\mathtt{c}$ for some full composition $\gamma_f$. We denote $\mathtt{T}\in \mathtt{X}$ the fact that~$\mathtt{T}$ is a structure agent included in~$\gamma_f$. 
Moreover, we denote $\#\mathtt{T}[\mathtt{X}]$ the number of occurrences of~$\mathtt{T}$ in $\gamma_f$. 
\item For a complex agent $\mathtt{X}=\gamma_f::\mathtt{c}$ where each item $x\in\mathtt{X}$ is an agent assigned to a compartment $\mathtt{c}$, we can use simplified notation that omits the compartment suffix `$::\mathtt{c}$' in individual agents of $\gamma_f$.
\end{itemize}
\end{notation}

Next we define structural equivalence of complex agents. We employ set-based approach to aggregate complex agents into equivalence classes. In particular, at that level we achieve commutativity and associativity of the operator `$.$'.  

\begin{defn}
Let $\mathtt{X},\mathtt{X}'$ be complex agents. We define \emph{structural equivalence} of \emph{complex agents} by claiming $\mathtt{X}\equiv\mathtt{X}'$ iff either of the following conditions holds:
\begin{enumerate}
\item There exist a compartment $\mathtt{c}$ and $n,n'\in\mathcal{N}_x$ such that $\mathtt{X}=n::\mathtt{c},\mathtt{X}'=n'::\mathtt{c}$ and $n=n'$.
\item If both $\mathtt{X},\mathtt{X}'$ are specified as full compositions then the following two conditions must be satisfied:

\begin{itemize}
\item for each $\mathtt{T}\in\mathtt{X}$ there exists $\mathtt{T}'\in\mathtt{X}'$ such that  $\mathtt{T}\equiv\mathtt{T}'$ and $\#\mathtt{T}[\mathtt{X}]=\#\mathtt{T}'[\mathtt{X}']$,
\item for each $\mathtt{T}'\in\mathtt{X}'$ there exists $\mathtt{T}\in\mathtt{X}$ such that  $\mathtt{T}'\equiv\mathtt{T}$ and $\#\mathtt{T}'[\mathtt{X}']=\#\mathtt{T}[\mathtt{X}]$.
\end{itemize}
\end{enumerate}
\end{defn}

An example of a complex agent is given in Table~\ref{table:caseExmp} where the given complex agent expression represents a large set of hexamers composed from $KaiC$ molecules each considered in arbitrary state.  

\begin{remark}
From now on, we always consider a \emph{lexicographically ordered agent} as a \emph{representative of a class of structurally equivalent agents}. Since agents are defined hierarchically, lexicographical order is applied recursively to all nested agents. This allows us to always have a clearly defined unique representative.
\end{remark}

\end{document}

\subsection{Signatures}

A complex agent is given either directly as an expression inductively built by applying coexistence operator `$.$' to atomic and structure agents or indirectly as a name referring to a separate set of definitions of complex agents (incorporated in the notion of agent signature). We use that approach because we do not want to overcomplicate complex agent expressions. 