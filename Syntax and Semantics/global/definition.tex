\documentclass{entcs}
\usepackage{prentcsmacro}
\usepackage{kpfonts}
\usepackage{hyperref}
\usepackage{algorithmicx}
\usepackage{graphicx}
\usepackage[noend]{algpseudocode}
\usepackage{mathabx}
\usepackage{leftidx}

\hyphenation{me-ta-bo-lism}

% global macro definitions
\newcommand{\TBD}{\textbf{TBD}~}
\newcommand{\spec}[1]{\texttt{#1}}
\renewcommand{\~}[0]{\texttildelow}
\renewcommand{\algorithmiccomment}[1]{\hfill\textit{\# #1}}
\newcommand*\arc{{\fontfamily{pbk}\fontseries{db}\selectfont+}}
\newcommand{\pipe}{{}{\scalebox{.6}{$|$}}{}}
\newcommand{\choice}{|}

\newtheorem{notation}[thm]{Notation}

\def\lastname{Troj\'{a}k et al.}

\begin{document}

% title page
\begin{frontmatter}
\title{Formal Biochemical Space Language}

\author{\normalsize
T. D\v{e}d, D. \v{S}afr\'anek, M. Troj\'ak, M. Klement, J. \v{S}alagovi\v{c}, L. Brim}
\address{Faculty of Informatics, Masaryk University\\
Brno, Czech Republic
}

\end{frontmatter}

\section{Biochemical Space Agents}

note - maybe all compartment could be removed from syntax of agents and added on level of rules

Let $\mathcal{N}_{a},~\mathcal{N}_{T},~\mathcal{N}_{x},~\mathcal{N}_{c},~\mathcal{N}_{s}$ be mutually exclusive finite sets of atomic, structure, complex, compartment, and state names, respectively. 

\subsection{Atomic agents}

Agents are defined hierarchically starting from \emph{atomic agents}. These agents are the smallest units in our hierarchy a represent ...

\begin{definition}
Atomic agent $\mathtt{A}$ is defined as triple ($\alpha, \delta, \mathtt{c}$) where $\alpha \in \mathcal{N}_{A}$ is name, $\delta \subseteq \mathcal{N}_{s}$ is finite non-empty set of states $\mathtt{s} \in \mathcal{N}_{s}$ in which the agent can occur, and $\mathtt{c} \in \mathcal{N}_{c}$ is physical compartment within which it is considered.
\end{definition}

\begin{definition}
Let $\mathtt{A},~\mathtt{A}'$ be atomic agents. We define the \emph{equivalence of atomic agents} by claiming $\mathtt{A}\equiv\mathtt{A}'$ whenever $\mathtt{\alpha} = \mathtt{\alpha}'$, $\mathtt{c} = \mathtt{c}'$ and $\delta = \delta'$.
\end{definition}

\begin{defn}
Let $\mathtt{A},\mathtt{A}'$ be atomic agents. We say agent $\mathtt{A}=(\alpha, \delta, \mathtt{c})$ is \emph{compatible with} agent $\mathtt{A}'=(\alpha', \delta', \mathtt{c}')$, written $\mathtt{A} \lhd \mathtt{A}'$, iff $\alpha = \alpha'$, $\mathtt{c} = \mathtt{c}'$, and $\delta \subseteq \delta'$. 
\end{defn}

\begin{notation}
{~}
\begin{itemize}
\item We denote $\mathtt{s}\in\delta$ the fact that $\mathtt{s}$ is included in the state signature~$\delta$.
\end{itemize}
\end{notation}

Atomic agent expressions have the following syntax:

\begin{center}
{\small
\hspace*{-1cm}\begin{tabular}{ ll ll ll ll }
 atomic agent expression & $\mathtt{A} ::= \alpha\{\delta\}::\mathtt{c}$ & states & $ \delta ::= s_1, s_2, ..., s_m$ \\
 name & $\alpha ::= n \in \mathcal{N}_{A}$  & state & $\mathtt{s}_i ::= n \in \mathcal{N}_{s}$\\
 compartment & $\mathtt{c} ::= n \in \mathcal{N}_{c}$\\
\end{tabular}
}
\end{center}

\begin{notation}
{~}
\begin{itemize}
\item Note that an atomic agent expression fully defines an atomic agent. However, an atomic agent might be written by more expressions due to order in enumerated states. We write $\Upsilon_\mathtt{A}$ as set of all possible atomic agent expressions created from atomic agent $\mathtt{A}$.
\end{itemize}
\end{notation}

\begin{theorem}
Since atomic agents are constructed from sets $\mathcal{N}_{A},~\mathcal{N}_{c}$, and $\mathcal{N}_{s}$, we can define universe of atomic agents $\mathcal{A} = \{ \mathtt{A}~|~\mathtt{A} = (\alpha, \delta, \mathtt{c}), \alpha~\in~\mathcal{N}_{A}, \delta~\subseteq~\mathcal{N}_{s},  \mathtt{c}~\in~\mathcal{N}_{c} \}$. 

\noindent Similarly, we can define universe of atomic agent expressions\\ $\mathcal{E}_\mathcal{A} = \{ \alpha\{\delta\}::\mathtt{c} ~|~ (\alpha, \delta, \mathtt{c}) \in \mathcal{A} \} $.
\end{theorem}

\subsection{Structure agents}

Next we proceed with defining \emph{structure agents}. A structure agent represents a biochemical object that is composed from several known atomic agents provided that we know that such a composition is abstract and not necessarily complete. To incorporate such an abstraction of biological structures into our language, a structure agent is defined to be labelled with a unique name and it is constructed only from atomic agents considered in the same physical compartment. 

A typical example of a structure agent is a protein where the atomic agents are individual amino acids that are of interest in the particular setting.

\begin{definition}
Structure agent $\mathtt{T}$ is defined as triple ($\tau, \gamma_p, \mathtt{c}$) where $\tau \in \mathcal{N}_{T}$ is name, $\gamma_p \subseteq \mathcal{A}$ is finite set of atomic agents $\mathtt{A} \in \mathcal{A}$ called \emph{partial composition}, and $\mathtt{c} \in \mathcal{N}_{c}$ is physical compartment within which it is considered. \textbf{Note that in partial composition cannot appear two atomic agents with same name.}
\end{definition}

\begin{notation}
{~}
\begin{itemize}
\item We denote $\mathtt{A} \in \mathtt{T}$ the fact that $\mathtt{A}$ is included in partial composition $\gamma_p$ of structure agent $\mathtt{T}$.
\item Note that a compartment of a structure agent is uniquely given by the compartment specified in its parts. We restrict ourselves to structure agents where all atomic agents in the partial composition have the same compartment. Assuming this restriction, we can shorten the notation by omitting compartments in the atomic agents of a partial composition. 
\end{itemize}
\end{notation}

\begin{defn}
Let $\mathtt{T},\mathtt{T}'$ be structure agents. We define the \emph{equivalence of structure agents} by claiming $\mathtt{T}\equiv\mathtt{T}'$ iff there exist $\tau,\tau',\gamma_p,\gamma'_p,\mathtt{c},\mathtt{c}'$ such that $\mathtt{T}=(\tau, \gamma_p, \mathtt{c}),\mathtt{T}'=(\tau', \gamma_p', \mathtt{c}')$, $\tau=\tau'$, $\mathtt{c} = \mathtt{c}'$, and $\gamma_p=\gamma'_p$.
\end{defn}

\begin{defn}
Let $\mathtt{T},\mathtt{T}'$ be structure agents. We say $\mathtt{T}$ \emph{is compatible with} $\mathtt{T}'$, written $\mathtt{T} \lhd \mathtt{T}'$, iff there exist $\tau,\tau',\gamma_p,\gamma'_p,\mathtt{c},\mathtt{c}'$ such that $\mathtt{T}=(\tau, \gamma_p, \mathtt{c}),\mathtt{T}'=(\tau', \gamma_p', \mathtt{c}')$, $\tau = \tau'$, $\mathtt{c} = \mathtt{c}'$, and for each atomic agent $\mathtt{A} \in \gamma_p$ there exists an atomic agent $\mathtt{A}' \in \gamma'_p$ such that $\mathtt{A}~\lhd~\mathtt{A}'$. 
\end{defn}

Structure agent expressions have the following syntax:

\begin{center}
{\small
\hspace*{-1cm}\begin{tabular}{ ll ll }
 structure agent expression & $\mathtt{T} ::= \tau(\gamma_p)::\mathtt{c}$\\
 structure name & $\tau ::= n \in \mathcal{N}_{T}$\\
 partial composition & $\gamma_p ::= \mathtt{A},~\gamma_p~\choice~\mathtt{A}$\\
\end{tabular}
}
\end{center}   

\begin{notation}
~
\begin{itemize}
\item Note that an structure agent expression fully defines a structure agent. However, a structure agent might be written by more expressions due to order in partial composition. We write $\Upsilon_\mathtt{T}$ as set of all possible structure agent expressions created from structure agent $\mathtt{T}$.
\end{itemize}
\end{notation}

\begin{theorem}
Since structure agents are constructed from sets $\mathcal{N}_{T},~\mathcal{N}_{c}$, and $\mathcal{A}$, we can define universe of structure agents $\mathcal{T} = \{ \mathtt{T}~|~\mathtt{T} = (\tau, \gamma_p, \mathtt{c}), \tau~\in~\mathcal{N}_{T}, \gamma_p~\subseteq~\mathcal{A},  \mathtt{c}~\in~\mathcal{N}_{c} \}$.

\noindent Similarly, we can define universe of structure agent expressions\\ $\mathcal{E}_\mathcal{T} = \{ \tau(\gamma_p)::\mathtt{c} ~|~ (\tau, \gamma_p, \mathtt{c}) \in \mathcal{T} \}$.
\end{theorem}

\subsection{Complex agents}

In the following we define the last step in the hierarchy of agents. In particular, we define \textit{complex agents}. A complex agent represents a non-trivial composite biochemical object that is (inductively) constructed from already known biological objects. In common rule-based languages this is typically defined by introducing some kind of bonds between individual biochemical objects. In BCS we abstract from detailed specification of bonds and we rather assume a complex as a coexistence of certain objects in a particular group. Such a group can be optionally referred to by a unique name. A complex agent is constructed from atomic and structure agents.

\begin{definition}
Complex agent $\mathtt{X}$ is defined as pair ($\gamma_f, \mathtt{c}$) where $\gamma_f \subseteq \mathcal{A} \cup \mathcal{T}$ is finite multiset of atomic and structure agents called \emph{full composition} and $\mathtt{c} \in \mathcal{N}_{c}$ is physical compartment within which it is considered.
\end{definition}

\begin{notation}
~
\begin{itemize}
\item  We denote $\mathtt{T} \in \mathtt{X}$ (resp. $\mathtt{A}\in \mathtt{X}$) the fact that~$\mathtt{T}$ (resp. $\mathtt{A}$) is a structure agent (resp. atomic agent) included in~$\gamma_f$. 
\item We denote $\mathtt{c}(\mathtt{X})$ the fact that $\mathtt{X}$ is in compartment $\mathtt{c}$.
\item Note that a compartment of a complex agent is uniquely given by the compartment specified in its parts. We restrict ourselves to complex agents where all atomic and structure agents in the full composition have the same compartment. Assuming this restriction, we can shorten the notation by omitting compartments in the atomic and structure agents of a full composition. 
\end{itemize}
\end{notation}

\begin{defn}
Let $\mathtt{X},\mathtt{X}'$ be structure agents. We define the \emph{equivalence of structure agents} by claiming $\mathtt{X} \equiv \mathtt{X}'$ iff there exist $\gamma_f,\gamma'_f, \mathtt{c},\mathtt{c}'$ such that $\mathtt{X}=(\gamma_f, \mathtt{c}),\mathtt{T}'=(\gamma_f', \mathtt{c}')$, $\mathtt{c} = \mathtt{c}'$, and $\gamma_f=\gamma'_f$.
\end{defn}

Complex agent expressions have the following syntax:

\begin{center}
{\small
\hspace*{-1 cm}\begin{tabular}{ ll ll }
 complex agent expression & $\mathtt{X}::=\gamma_f::\mathtt{c}$\\
 full composition & $\gamma_f ::= \mathtt{T}.\mathtt{T}~|~\mathtt{A}.\mathtt{A}~|~\mathtt{T}.\gamma_{f}~|~\mathtt{A}.\gamma_{f}$
\end{tabular}
}
\end{center}

\begin{notation}
~
\begin{itemize}
\item Note that an complex agent expression fully defines a complex agent. However, a complex agent might be written by more expressions due to order in full composition. We write $\Upsilon_\mathtt{X}$ as set of all possible complex agent expressions created from complex agent $\mathtt{X}$.
\end{itemize}
\end{notation}

\begin{theorem}
Since complex agents are constructed from sets $\mathcal{N}_{x},~\mathcal{N}_{c}$, $\mathcal{A}$, and $\mathcal{T}$, we can define universe of complex agents $\mathcal{X} = \{ \mathtt{X}~|~\mathtt{X} = (\gamma_f, \mathtt{c}), \gamma_f \subseteq \mathcal{A} \cup \mathcal{T},  \mathtt{c}~\in~\mathcal{N}_{c} \}$.

\noindent Similarly, we can define universe of complex agent expressions\\ $\mathcal{E}_\mathcal{X} = \{ \gamma_f::\mathtt{c} ~|~ (\gamma_f, \mathtt{c}) \in \mathcal{X} \}$.
\end{theorem}

\subsection{Summary}

\begin{definition}
We define universe of all agents $\mathcal{U} = \mathcal{A} \cup \mathcal{T} \cup \mathcal{X}.$

Finally, we define universe of all agent expressions $\mathcal{E} = \mathcal{E}_\mathcal{A} \cup \mathcal{E}_\mathcal{T} \cup \mathcal{E}_\mathcal{X} \cup \{\varepsilon\}$.
\end{definition}

\begin{notation}
~
\begin{itemize}
\item Since an expression uniquely defines an agent, we can use compatibility between agent expressions.
\end{itemize}
\end{notation}

\section{Biochemical Space Rules}

%\begin{definition}
%Rule $\mathtt{R}$ is defined as pair ($\lambda, \rho$) where $\lambda, \rho \subseteq \mathcal{E}$ and $\lambda$ is \emph{left-hand-side} of the rule and $\rho$ is \emph{right-hand-side} of the rule. Both $\lambda$ and $\rho$ are ordered sets where order is given by rule expression and at most one of them might be $\varepsilon$.
%\end{definition}

Rule is an expression defined from agent expressions $\mathcal{E}$ by the following grammar:

\begin{center}
{\small
\hspace*{-1cm}\begin{tabular}{ l l }
 rule expression & $\mathtt{R} ::= \lambda ~\Rightarrow~ \rho ~|~ \lambda ~\Rightarrow ~|~ \Rightarrow~ \rho $\\
  & $\lambda ::= \Gamma$\\
  & $\rho ::= \Gamma$\\
 agent enumeration & $\Gamma ::= \varphi~ +~\Gamma ~|~ \varphi$\\
 rule agent & $\varphi :: = \mathtt{A}~|~\mathtt{T}~|~\mathtt{X}$\\
\end{tabular}
}
\end{center}

\begin{defn}
Let $\mathtt{R},\mathtt{R}'$ be rules. We define the \emph{structural equivalence of rules} by claiming $\mathtt{R} \equiv \mathtt{R}'$ iff there exist $\lambda, \lambda', \rho, \rho', \nu, \nu'$ such that $\mathtt{R}=\lambda ~\Rightarrow~ \rho$, $\mathtt{R}'=\lambda' ~\Rightarrow~ \rho'$, $\lambda=\lambda'$, and $\rho=\rho'$.
\end{defn}

\begin{notation}
~
\begin{itemize}
\item Agent enumeration $\Gamma$ is an expression of form $\varphi_1 + \varphi_2 + ... \varphi_n$. From now, when referencing it, it might be treated as tuple $(\varphi_1, \varphi_2, ..., \varphi_n)$.
\end{itemize}
\end{notation}

\begin{theorem}
We define universe of rule expressions\\ $\mathcal{E}_\mathcal{R} = \{ \lambda ~\Rightarrow~ \rho \} \cup \{ \lambda ~\Rightarrow  \} \cup \{ \Rightarrow~ \rho \}$.
\end{theorem}

\section{Semantics}

How is semantics being applied on a model does not matter now. Briefly, using an algorithm, we choose a solution from given state and apply a rule on this solution. This step is quite forward and does not need any deeper investigation. What has to be defined is application of the rule on the solution.

Let $\beta$ be an agent, $\mathtt{R} \in \mathcal{R}$ be a rule of form $\lambda \Rightarrow \rho$ where $\lambda$ and $\rho$ are agent enumerations, and $\mathtt{S} \subseteq \mathcal{U}$ be a solution. 

\begin{definition}
For a given solution $\mathtt{S}$, we denote $\Upsilon_\mathtt{S}$ as set of all possible agent enumerations constructed from $\Upsilon_\mathtt{A}$, $\Upsilon_\mathtt{T}$, and $\Upsilon_\mathtt{X}$ for particular agents in $\mathtt{S}$. We denote $\mathtt{S}_\Upsilon$ an element in $\Upsilon_\mathtt{S}$.
\end{definition}

\begin{definition}
$ $ \\
\textbf{Matching} is a function $\models ~: \Upsilon_\mathtt{S}~\bigtimes~\lambda~\rightarrow~\Upsilon_\mathtt{S}^{\lhd}$ defined structurally:

\begin{itemize}
	\item $\mathtt{A}_\mathtt{S} \models \mathtt{A}_{\lambda} \Leftrightarrow \mathtt{A}_\mathtt{S} \lhd \mathtt{A}_{\lambda}$,
	\item $\mathtt{T}_\mathtt{S} \models \mathtt{T}_{\lambda} \Leftrightarrow \mathtt{T}_\mathtt{S} \lhd \mathtt{T}_{\lambda}$,
	\item $\mathtt{X}_\mathtt{S} \models \mathtt{X}_{\lambda} \Leftrightarrow \mathtt{c}(\mathtt{X}_\mathtt{S}) = \mathtt{c}(\mathtt{X}_{\lambda})$ and $\forall \beta_1, \beta_2, ..., \beta_n \in \mathtt{X}_\mathtt{S} $ \& $\forall \beta_1', \beta_2', ..., \beta_n' \in \mathtt{X}_{\lambda}$ : $\beta_i \lhd \beta_i'$.
	
\end{itemize}
\noindent The resulting set $\Upsilon_\mathtt{S}^{\lhd}$ contains all those expressions from $\Upsilon_\mathtt{S}$ which match \emph{left-hand-side} $\lambda$ of the rule $\mathtt{R}$.

\end{definition}

\begin{definition}
$ $ \\
\textbf{Replacement} is a function $\Upsilon_\mathtt{S}^{\lhd}~\times~\rho~\rightarrow~\Upsilon_\mathtt{R}$, where $\Upsilon_\mathtt{R}$ is set of agent enumerations created according to six replacement categories:

\begin{enumerate}
\item \textit{state change }:

\begin{itemize}
\item $ \pmb{\alpha_\mathtt{S}\{p\}::c[\alpha_\rho\{t\}::c] = \alpha_\mathtt{\mathtt{S}}\{t\}::c} $

where $ \alpha_\mathtt{S} = \alpha_\rho $ and $p \neq t$.

\item $ \pmb{\mathtt{T}_\mathtt{S}::c[\mathtt{T}_\rho::c] = \mathtt{T}_\mathtt{R}::c}$ 

$\mathtt{T}_\mathtt{R} = \mathtt{A}_{\mathtt{R}_1}, \mathtt{A}_{\mathtt{R}_2}, ..., \mathtt{A}_{\mathtt{R}_n} : \mathtt{A}_{\mathtt{R}_1} = \alpha_\mathtt{S}\{p\}::c[\alpha_\rho\{t\}::c]$

and $\alpha_\rho\{t\} \in \mathtt{T}_\rho, ~\alpha_\mathtt{S}\{p\} \in \mathtt{T}_\mathtt{S} $ and $\alpha_\mathtt{S} = \alpha_\rho$.

\item $ \pmb{\mathtt{X}_\mathtt{S}::c[\mathtt{X}_\rho::c] = \mathtt{X}_\mathtt{R}::c}$

$ \mathtt{X}_\mathtt{R} = \beta_1.\beta_2. ~...~ .\beta_n \pmb{:} \forall y_1, y_2, ..., y_n \in \mathtt{X}_\mathtt{S}~and~z_1, z_2, ..., z_n \in \mathtt{X}_\rho \pmb{:}$

$\beta_i = y_i::c~[z_i::c].$

\end{itemize}

\item \textit{complex formation}:

\begin{itemize}
\item $\pmb{\mathtt{S}_\Upsilon[\mathtt{X}_\rho::c] = \mathtt{X}_\mathtt{R}::c}$

$\mathtt{X}_\mathtt{R} = \beta_1.\beta_2. ~...~ .\beta_n \pmb{:} \forall \beta_i::c \in \mathtt{S}_\Upsilon.$

\end{itemize}

\item \textit{complex dissociation}:

\begin{itemize}
\item $ \pmb{\mathtt{X}_\mathtt{S}::c[\rho] = \mathtt{R}_\Upsilon}$ 

$\rho = \beta_{\rho_1}::c + \beta_{\rho_2}::c + ~...~ + {\rho_n}::c$ and $\mathtt{X}_\mathtt{S} = \beta_{\mathtt{S}_1}.\beta_{\mathtt{S}_2}. ~...~ .{\mathtt{S}_n}::c$.

$\mathtt{R}_\Upsilon = \beta_{\mathtt{R}_1}::c + \beta_{\mathtt{R}o_2}::c + ~...~ + {\mathtt{R}_n}::c \pmb{:}$ $ \beta_{\mathtt{R}_i} = \beta_{\mathtt{S}_i} \Leftrightarrow \beta_{\mathtt{S}_i} \models \beta_{\rho_i}.$

\end{itemize}

\item \textit{degradation}
\begin{itemize}
\item $  \pmb{\mathtt{S}_\Upsilon[\emptyset] = \emptyset} $
\end{itemize}

\item \textit{translation}
\begin{itemize}
\item $ \pmb{\emptyset[\rho] = \rho} $
\end{itemize}

\item \textit{transport}:
\begin{itemize}
\item $ \pmb{\beta_\mathtt{S}::c_\mathtt{S}[\beta_\rho::c_\rho] = \beta_\mathtt{S}::c_\rho} $

where $\beta_\mathtt{S} = \beta_\rho$ and $ c_\mathtt{S} \neq c_\rho $.
\end{itemize}

\end{enumerate}
\end{definition}

\subsection{Syntactic extensions}

We define several syntactic extensions for better readability of the rules. Note that each rule in a extended form can be always translated to basic form defined above.

\textbf{signatures}

for atomic its maximal state set

for structure its neccessary composition

for complex its substitution

\textbf{direction}

\textbf{Stoichiometry}

\textbf{Variables}

\textbf{Locations}

\begin{center}
{\small
\hspace*{-1cm}\begin{tabular}{ l l }
 rule expression & $\mathtt{R} ::= \lambda ~\odot~ \rho~\nu ~|~ \lambda ~\odot~\nu ~|~ \odot~ \rho~\nu $\\
  & $\lambda ::= \Gamma$\\
  & $\rho ::= \Gamma$\\
  & $\nu ::=~ ;~\upsilon ~|~ \varepsilon$\\
 direction & $\odot ::=~ \Rightarrow~|~\Leftrightarrow $\\
  & $\Gamma ::= \zeta~\varphi~ +~\Gamma ~|~ \zeta~\varphi$\\
 stoichiometry & $\zeta ::= n \in \mathbb{N}^+$\\
 rule agent & $\varphi :: = \varphi_1~|~\varphi_2~|~\varphi_1::\xi~|~\varphi_2::\xi'~|~\xi'$\\
  & $\varphi_1  ~::=~  ?\nu_1~|~?\nu_2~|~?\nu_3~|~\mathtt{A}::~?\nu_2~|~?\nu_1::\mathtt{T}$\\
  & $\xi' ~::=~ \mathtt{X}::\xi'~|~?\nu_3::\xi~|~?\nu_3~|~\xi$\\
  & $\varphi_2  ~::=~  \mathtt{A}~|~\mathtt{T}~|~\mathtt{X}~|~\mathtt{A}::\mathtt{T}$\\
  & $\xi  ~::=~  \mathtt{X}::\xi~|~\mathtt{X}$\\
 variable & $\upsilon ::=~?\nu_1=\{\phi_1\}~|~?\nu_2=\{\phi_2\}~|~?\nu_3=\{\phi_3\}$\\
  & $\phi_1 ::= \mathtt{A},~\phi_1~|~\mathtt{A}$\\
  & $\phi_2 ::= \mathtt{T},~\phi_2~|~\mathtt{T}$\\
  & $\phi_3 ::= \mathtt{X},~\phi_3~|~\mathtt{X}$\\
\end{tabular}
}
\end{center}

\begin{definition}
Let $\varphi$ be a rule agent that appears in a rule $\mathtt{R}$. The \emph{rule agent} $\varphi$ is \emph{well-formed} iff the following constrains are satisfied:

\begin{enumerate}
 \item If $\mathtt{A}::\mathtt{T}$ is a subexpression of $\varphi$ and $\mathtt{T} = \tau(\gamma_p)::\mathtt{c}$ then exists $\mathtt{A}' \in \gamma_p$ such that $\mathtt{A} \lhd \mathtt{A}'$.

\item If $\mathtt{T}::\mathtt{X}$ is a subexpression of $\varphi$ then exists $\mathtt{T}' \in \mathtt{X}$ such that $\mathtt{T} \lhd \mathtt{T}'$.

\item If $\mathtt{A}::\mathtt{X}$ is a subexpression of $\varphi$ then exists $\mathtt{A}' \in \mathtt{X}$ such that $\mathtt{A} \lhd \mathtt{A}'$.

\item If $\mathtt{X}'::\mathtt{X}$ is a subexpression of $\varphi$ then exists $\mathtt{X}'' \in \mathtt{X}$ such that $\mathtt{X}' \lhd \mathtt{X}''$.

\end{enumerate}

\end{definition}

\begin{definition}[Rule Flattening]
Let $(\Sigma_\tau,\Sigma_x)$ be a signature and $\mathtt{R}$ a set of rules. Every rule $r\in\mathtt{R}$ that includes some \emph{nested} rule agents $\varphi$ can be reduced to a rule $r'\in\mathtt{R}$ where every rule agent $\varphi$ is in \emph{basic} form. For every subexpression $\beta$ in every rule agent $\varphi$ in $r$, the reduction is done recursively by replacing $\beta$ with $\beta'$ in the following way:
\begin{enumerate}
\item If $\beta=\mathtt{A}::\mathtt{T}$ where $\mathtt{T}=\tau(\gamma_p)$ for some $\tau,\gamma_p$ then there must exist $\mathtt{A}'\in\gamma_p$ such that $\mathtt{A}\lhd\mathtt{A}'$. Then we set $\beta'={\tau}(\gamma'_p)$ where $\gamma'_p$ is constructed from $\gamma_p$ by replacing $\mathtt{A}'\in\gamma_p$ with $\mathtt{A}$.
\item If $\beta=\mathtt{T}::\mathtt{X}$ where $\mathtt{X}=\gamma_f$ then there must exist $\mathtt{T}'\in\gamma_f$ such that $\mathtt{T}\lhd\mathtt{T}'$. Then we set $\beta'=\gamma'_f$ where $\gamma'_f$ is constructed from $\gamma_f$ by replacing $\mathtt{T}'\in\gamma_f$ with $\mathtt{T}$. 
\item If $\beta=\mathtt{A}::\mathtt{X}$ where $\mathtt{X}=\gamma_f$ then there must exist $\mathtt{A}'\in\gamma_f$ such that $\mathtt{A}\lhd\mathtt{A}'$. Then we set $\beta'=\gamma'_f$ where $\gamma'_f$ is constructed from $\gamma_f$ by replacing $\mathtt{A}'\in\gamma_f$ with $\mathtt{A}$. 
\item If $\beta=\mathtt{X}'::\mathtt{X}$ where $\mathtt{X}=\gamma_f$ and $\mathtt{X}'= \gamma'_f$ then for every $\mathtt{T} \in \gamma_f$ (resp. $\mathtt{A} \in \gamma_f$) there must exist $\mathtt{T}' \in \gamma'_f$ (resp. $\mathtt{A}' \in \gamma'_f$) such that $\mathtt{T} \lhd \mathtt{T}'$ (resp. $\mathtt{A} \lhd \mathtt{A}'$). Then we set $\beta'=\gamma_{f'}$ where $\gamma_{f'}$ is constructed from $\gamma_f$ by replacing all $\mathtt{T} \in \gamma_f$ (resp. $\mathtt{A} \in \gamma_f$) with appropriate $\mathtt{T}' \in \gamma'_f$  (resp. $\mathtt{A}' \in \gamma'_f$).
\end{enumerate} 
\end{definition}



\end{document}
