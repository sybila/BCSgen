\documentclass[12pt]{article}
\usepackage[slovak]{babel}
\usepackage[utf8x]{inputenc}
\usepackage{kpfonts}
\usepackage{hyperref}
\usepackage{amsthm}
\newtheorem{mydef}{Definition}

\newcommand{\mysection}[1]{{\newpage\centering\Large\textbf{#1}\\}\normalsize\vspace{0.5cm}}
\newcommand{\mysmallsection}[1]{\vspace{0.5cm}\large\textbf{#1}\normalsize\vspace{0.5cm}}
\newcommand{\mydefinition}[1]{\vspace{0.5cm}\large\textbf{Definition} \normalsize{#1}\vspace{0.5cm}}

\begin{document}

\mysection{Extended BCSL syntax}

With this definition, we are allowed to use `::' operator also between two complexes. For this reason, we can extend the syntax of the language as following:

\begin{center}
{\small
\hspace*{-1cm}\begin{tabular}{ ll ll }
 rules& $\mathtt{R} ::= \emptyset ~|~ r, \mathtt{R} $\\
 rule equation & $r ::= \Gamma ~\odot~\Gamma$\\
 direction & $\odot ::=~ \Rightarrow~|~\Leftrightarrow $\\
 rule expression & $\Gamma ::= \emptyset~|~\varrho~\epsilon::\mathtt{c}~|~ \varrho~\epsilon::\mathtt{c}~ +~\Gamma$\\
 stoichiometry & $\varrho ::= n \in \mathbb{N}^+$\\
 rule expression item & $\epsilon :: = \epsilon_1~|~\epsilon_1::\xi$\\
 rule agent & $\epsilon_1  ~::=~  \mathtt{a}~|~\mathtt{T}~|~\mathtt{X}~|~\mathtt{a}::\mathtt{T}$\\
 \textbf{nested complex} & $\xi  ~::=~  \mathtt{X}::\xi~|~\mathtt{X}$
\end{tabular}
}
\end{center}

Extended complex agent definition:

\begin{center}
{\small
\hspace*{-1 cm}\begin{tabular}{ ll ll }
 complex agent & $\mathtt{X}::=\gamma_f::\mathtt{c}~|~n\in{N}_{x}::\mathtt{c}$\\
 full composition & $\gamma_f ::= \mathtt{T}.\mathtt{T}~|~\mathtt{a}.\mathtt{a}~|~\mathtt{T}.\gamma_{f}~|~\mathtt{a}.\gamma_{f}$
\end{tabular}
}
\end{center}

\begin{mydef}
Let $\mathtt{X}=\gamma_f::\mathtt{c}$ for some full composition $\gamma_f$. We denote $\mathtt{T}\in \mathtt{X}$ (resp. $\mathtt{a}\in \mathtt{X}$) the fact that~$\mathtt{T}$ (resp. $\mathtt{a}$) is a structure agent (resp. atomic agent) included in~$\gamma_f$ (resp. in~$\gamma_p$). 
Moreover, we denote $\#\mathtt{T}[\mathtt{X}]$ (resp. $\#\mathtt{a}[\mathtt{X}]$) the number of occurrences of~$\mathtt{T}$ (resp. $\mathtt{a}$) in $\gamma_f$. 
\end{mydef}

\begin{mydef}
Let $\mathtt{X}=\gamma_f::\mathtt{c}$ for some full composition $\gamma_f$ and $\mathtt{X}'=\gamma'_f::\mathtt{c}$ for some full composition $\gamma'_f$. We denote $\mathtt{X} \in \mathtt{X}'$ the fact that each structure agent $\mathtt{T}$ (resp. atomic agent $\mathtt{a}$) $\in \gamma_f$ is included in~$\gamma'_f$, i.e. $\mathtt{X}$ is included in $\mathtt{X}'$.
\end{mydef}

\begin{mydef}
Let $\mathtt{X},\mathtt{X}'$ be complex agents. We define \emph{structural equivalence} of \emph{complex agents} by claiming $\mathtt{X}\equiv\mathtt{X}'$ iff either of the following conditions holds:
\begin{enumerate}
\item There exist a compartment $\mathtt{c}$ and $n,n'\in\mathcal{N}_x$ such that $\mathtt{X}=n::\mathtt{c},\mathtt{X}'=n'::\mathtt{c}$ and $n=n'$.
\item If both $\mathtt{X},\mathtt{X}'$ are specified as full compositions then the following four conditions must be satisfied:

\begin{itemize}
\item for each $\mathtt{T}\in\mathtt{X}$ there exists $\mathtt{T}'\in\mathtt{X}'$ such that $\mathtt{T}\equiv\mathtt{T}'$ and $\#\mathtt{T}[\mathtt{X}]=\#\mathtt{T}'[\mathtt{X}']$,
\item for each $\mathtt{T}'\in\mathtt{X}'$ there exists $\mathtt{T}\in\mathtt{X}$ such that $\mathtt{T}'\equiv\mathtt{T}$ and $\#\mathtt{T}'[\mathtt{X}']=\#\mathtt{T}[\mathtt{X}]$,
\item for each $\mathtt{a}\in\mathtt{X}$ there exists $\mathtt{a}'\in\mathtt{X}'$ such that $\mathtt{a}\equiv\mathtt{a}'$ and $\#\mathtt{a}[\mathtt{X}]=\#\mathtt{a}'[\mathtt{X}']$,
\item for each $\mathtt{a}'\in\mathtt{X}'$ there exists $\mathtt{a}\in\mathtt{X}$ such that $\mathtt{a}'\equiv\mathtt{a}$ and $\#\mathtt{a}'[\mathtt{X}']=\#\mathtt{a}[\mathtt{X}]$.
\end{itemize}

\end{enumerate}
\end{mydef}

\begin{mydef}
Let $\mathtt{X},\mathtt{X}'$ be complex agents. We say $\mathtt{X}$ \emph{is compatible} with $\mathtt{X}'$, written $\mathtt{X} \lhd \mathtt{X}'$, iff $\mathtt{X} \equiv \mathtt{X}'$ or iff for each $\mathtt{T} \in \mathtt{X}$ (resp. $\mathtt{a} \in \mathtt{X}$)  there exist unique $\mathtt{T}' \in \mathtt{X}'$ (resp. $\mathtt{a}' \in \mathtt{X}'$) such that $\mathtt{T} \lhd \mathtt{T}'$ (resp. $\mathtt{a} \lhd \mathtt{a}'$).
\end{mydef}

\begin{mydef}
Let $\epsilon$ be a rule expression item that appears in a rule $r\in\mathtt{R}$. The \emph{rule expression} $\epsilon$ is \emph{well-defined} iff the following constrains are satisfied:

\begin{enumerate}
 \item If $\mathtt{a}::\tau(\gamma_p)$ is a subexpression of $\epsilon$ for some $\mathtt{a},\tau,\gamma_p$ then there must exist $\mathtt{a}' \in \gamma_p$ such that $\mathtt{a} \lhd \mathtt{a}'$.

\item If $\mathtt{T}::\mathtt{X}$ is a subexpression of $\epsilon$ for some $\mathtt{T},\mathtt{X}$ then there must exist $\mathtt{T}' \in \mathtt{X}$ such that $\mathtt{T} \lhd \mathtt{T}'$.

\item If $\mathtt{a}::\mathtt{X}$ is a subexpression of $\epsilon$ for some $\mathtt{a},\mathtt{X}$ then there must exist $\mathtt{a}' \in \mathtt{X}$ such that $\mathtt{a} \lhd \mathtt{a}'$.

\item If $\mathtt{X}'::\mathtt{X}$ is a subexpression of $\epsilon$ for some $\mathtt{X}',\mathtt{X}$ then there must exist $\mathtt{X}'' \in \mathtt{X}$ such that $\mathtt{X}' \lhd \mathtt{X}''$.

\end{enumerate}
 
If one of above subexpressions is inside a rule expression then it is called \emph{nested} rule agent else it is a \emph{basic} rule agent.
\end{mydef}

\begin{mydef}[Rule Flattening]
Let $(\Sigma_\tau,\Sigma_x)$ be a signature and $\mathtt{R}$ a set of rules. Every rule $r\in\mathtt{R}$ that includes some \emph{nested} rule agents $\epsilon$ can be reduced to a rule $r'\in\mathtt{R}$ where every rule agent $\epsilon$ is in \emph{basic} form. For every subexpression $\beta$ in every rule agent $\epsilon$ in $r$, the reduction is done recursively by replacing $\beta$ with $\beta'$ in the following way:
\begin{enumerate}
\item If $\beta=\mathtt{a}::\mathtt{T}$ where $\mathtt{T}=\tau(\gamma_p)$ for some $\tau,\gamma_p$ then there must exist $\mathtt{a}'\in\gamma_p$ such that $\mathtt{a}\lhd\mathtt{a}'$. Then we set $\beta'={\tau}(\gamma'_p)$ where $\gamma'_p$ is constructed from $\gamma_p$ by replacing $\mathtt{a}'\in\gamma_p$ with $\mathtt{a}$.
\item If $\beta=\mathtt{T}::\mathtt{X}$ where $\mathtt{X}=\gamma_f$ then there must exist $\mathtt{T}'\in\gamma_f$ such that $\mathtt{T}\lhd\mathtt{T}'$. Then we set $\beta'=\gamma'_f$ where $\gamma'_f$ is constructed from $\gamma_f$ by replacing $\mathtt{T}'\in\gamma_f$ with $\mathtt{T}$. 
\item If $\beta=\mathtt{a}::\mathtt{X}$ where $\mathtt{X}=\gamma_f$ then there must exist $\mathtt{a}'\in\gamma_f$ such that $\mathtt{a}\lhd\mathtt{a}'$. Then we set $\beta'=\gamma'_f$ where $\gamma'_f$ is constructed from $\gamma_f$ by replacing $\mathtt{a}'\in\gamma_f$ with $\mathtt{a}$. 
\item If $\beta=\mathtt{X}'::\mathtt{X}$ where $\mathtt{X}=\gamma_f$ and $\mathtt{X}'= \gamma'_f$ then for every $\mathtt{T} \in \gamma_f$ (resp. $\mathtt{a} \in \gamma_f$) there must exist $\mathtt{T}' \in \gamma'_f$ (resp. $\mathtt{a}' \in \gamma'_f$) such that $\mathtt{T} \lhd \mathtt{T}'$ (resp. $\mathtt{a} \lhd \mathtt{a}'$). Then we set $\beta'=\gamma_{f'}$ where $\gamma_{f'}$ is constructed from $\gamma_f$ by replacing all $\mathtt{T} \in \gamma_f$ (resp. $\mathtt{a} \in \gamma_f$) with appropriate $\mathtt{T}' \in \gamma'_f$  (resp. $\mathtt{a}' \in \gamma'_f$).
\end{enumerate} 
\end{mydef}


\noindent 
\framebox{
\begin{minipage}{1.05\textwidth}
The extended syntax with variables is the following: 

\begin{center}
{\small
\hspace*{-1cm}\begin{tabular}{ l l }
 rules& $\mathtt{R} ::= r, \mathtt{R} ~|~ r $\\
 rule equation & $r ::= \zeta ~|~ \zeta~;~\upsilon$\\
 rule sides & $\zeta ::= \Gamma~\odot~\Gamma ~|~ \Gamma~\odot ~|~ \odot~\Gamma$\\
 direction & $\odot ::=~ \Rightarrow~|~\Leftrightarrow $\\
 rule expression & $\Gamma ::= \varrho~\epsilon::\mathtt{c}~ +~\Gamma ~|~ \varrho~\epsilon::\mathtt{c}$\\
 stoichiometry & $\varrho ::= n \in \mathbb{N}^+$\\
 rule expression item & $\epsilon :: = \epsilon_1~|~\epsilon_2~|~\epsilon_1::\xi~|~\epsilon_2::\xi'~|~\xi'$\\
 extended rule agent & $\epsilon_1  ~::=~  ?\nu_1~|~?\nu_2~|~?\nu_3~|~\mathtt{a}::~?\nu_2~|~?\nu_1::\mathtt{T}$\\
 extended nested complex & $\xi' ~::=~ \mathtt{X}::\xi'~|~?\nu_3::\xi~|~?\nu_3~|~\xi$\\
 rule agent & $\epsilon_2  ~::=~  \mathtt{a}~|~\mathtt{T}~|~\mathtt{X}~|~\mathtt{a}::\mathtt{T}$\\
 nested complex & $\xi  ~::=~  \mathtt{X}::\xi~|~\mathtt{X}$\\
 variable & $\upsilon ::=~?\nu_1=\{\phi_1\}~|~?\nu_2=\{\phi_2\}~|~?\nu_3=\{\phi_3\}$\\
 atomic variable value & $\phi_1 ::= \mathtt{a},~\phi_1~|~\mathtt{a}$\\
 structure variable value & $\phi_2 ::= \mathtt{T},~\phi_2~|~\mathtt{T}$\\
 complex variable value & $\phi_3 ::= \mathtt{X},~\phi_3~|~\mathtt{X}$\\
\end{tabular}
}
\end{center}
\end{minipage}
}

\vspace{0.5cm}
If a variable is used, then it has to be involved in both sides of the rule (if they exist) and also in `; $\upsilon$' part.

\mysection{Border between human-readable\\ and executable part of BCSL}

The main difference between these two parts of BCSL is that the executable part can have operational semantics while human-readable cannot. For operational semantics there must be clear relations between both sides of a rule. This can't be achieved in human-readable part because it represents abstract chemical reactions. 

The executable rule can be in one of three reaction types:

\begin{itemize}
\item state change
	\begin{itemize}	
		\item one agent is changing its state (or states)
		\item number of reactants on both sides of the rule cannot change
	\end{itemize}
\item complex formation/dissociation 
	\begin{itemize}
		\item the result of the rule must be creation of one complex (or dissociation of one complex)
		\item number of structure agents on both sides must be equal
	\end{itemize} 
\item degradation/translation 
	\begin{itemize}
		\item similar to complex formation/dissociation -- difference is empty rule side instead of a complex
		\item only object agents can be translated
		\item degradation can be specified also for classes
	\end{itemize} 
\end{itemize}

\noindent 
\framebox{
\begin{minipage}{1.05\textwidth}
\textbf{To determine whether a rule is executable, it has to meet these conditions:}

\begin{enumerate}
\item the rule has to be in \textbf{flattened form} - it means each agent involved in the rule is in basic form, i.e. it is atomic, structure or complex agent (the operator `::' can be used only in sense of compartmentalisation)
\item the number of interacting molecules must be \textbf{fixed} ($\#(LHS)~=~\#(RHS)$). Exceptions:
	\begin{enumerate}
		\item translation/degradation - one of the rule sides must be empty
		\item (de)complexation - number of new agents is different BUT number of all structure (resp. atomic) agents must be fixed
	\end{enumerate}
\item according to the reaction types of rules, the rule must be \textbf{unary operation} - only one agent can change in one step
\item all complexes must be \textbf{expanded} according to its specification
\end{enumerate}
If at least one of the conditions is not satisfied, then the rule is only in human-readable form.
\end{minipage}
}

\vspace*{1cm}
Then, if a human-readable rule is defined, exact mapping to executable rule(s) must be specified. The following steps are applied:

\begin{enumerate}
\item flattering can be resolved \textbf{automatically} - each rule expression item can be reduced to flattered form
\item if number is not \textbf{fixed}, then do following until it is not fixed (user's specification):
	\begin{enumerate}
		\item some of the agents put to \textit{MODIFIERS} field or
		\item remove some of the agents 
		\item (advanced: or each item could have a tag$^1$ which connects items from both sides and expresses exact rule behaviour)
	\end{enumerate}
\item if more than one operations is applied in the rule (combination of more reaction types or its multiplication), then the rule must be \textbf{expanded into more rules} (user's specification):
	\begin{enumerate}
		\item then, each created rule represents unary operation
		\item i.e., one human-readable rule triggers more executable rules
		\item note \textit{state change} reaction type can include change of more states at once
	\end{enumerate}
\item complexes can be expanded \textbf{automatically}
\end{enumerate}

\mysmallsection{Examples:}
\begin{itemize}
\item $S\{u\}::KaiC::KaiC2::cyt \Leftrightarrow S\{p\}::KaiC::KaiC2::cyt$
	\begin{itemize}
		\item An KaiC protein involved in KaiC2 dimer can change its phosphorylation state on serine residue
		\item \textbf{human-readable} rule, because:
			\begin{itemize}
				\item rule expression item are \textbf{not} in flattered form
				\item complexes are \textbf{not} expanded
			\end{itemize}
		\item both deficiencies can be fixed automatically
		\item resulting rule:\\
	$KaiC(S\{u\}).KaiC::cyt \Leftrightarrow KaiC(S\{p\}).KaiC::cyt$
	\end{itemize}
\item $FRS(T\{p\})::cyt + GS(T\{u\})::cyt \Leftrightarrow FRS(T\{p\}).GS(T\{u\})::cyt$
	\begin{itemize}
		\item threonine phosphorylated FRS enzyme and threonine unphosphorylated GS enzyme forms a complex
		\item \textbf{executable} rule, all conditions are satisfied
	\end{itemize}
\item $ps2(oec\{3+\}|yz\{+\})::tlm \Leftrightarrow ps2(oec\{4+\}|yz\{n\})::tlm$
	\begin{itemize}
		\item oxidation of S3-state of the oxygen-evolving complex OEC\{3+\} by Yz\{+\} in PSII RC
		\item it is allowed to change more states in one structure (complex) agent at once (it is considered as \textbf{one} change of ps2)
		\item \textbf{executable} rule, all conditions are satisfied
	\end{itemize}
	\newpage
\item {\small $pq\{n\}::tlm + 2 h\{+\}::tlm + bhc\{2-\}::cytb6f::tlm \\ \hspace*{1.9cm} \Leftrightarrow pqh2::tlm + bhc\{n\}::cytb6f::tlm$}
	\begin{itemize}
		\item \textbf{human-readable} rule, because:
			\begin{itemize}
				\item there are more operations happening at once
				\item uncertain $pqh2$ agent - might be complex, but it cannot be constructed from atomic agents
				\item (actually not even human-readable...)
			\end{itemize}
		\item possible \textbf{solution} in old BCS definition:
			\begin{itemize}
				\item there must be defined structure agent $pqh2(pq|h1|h2)$ where $h1$ and $h2$ can have $+$ or $n$ state and $pq$ can be omitted since it does not change its state during the rule
				\item expand the rule to following two executable rules:\vspace*{0.2cm}\\
				$> pqh2::tlm + cytb6f::tlm \Leftrightarrow pqh2.cytb6f::tlm $\vspace*{0.2cm}\\
				$> pqh2(h1\{+\}|h2\{+\}).cytb6f(bhc\{2-\})::tlm \\ \hspace*{.9cm} \Leftrightarrow pqh2(h1\{n\}|h2\{n\}).cytb6f(bhc\{n\})::tlm$
			\end{itemize}
		\item possible \textbf{solution} in \textbf{new} BCS definition:
			\begin{itemize}
			    \item there must be defined complex agent $pqh2 = pq.h.h$
				\item expand the rule to following two executable rules:\vspace*{0.2cm}\\
				$> pq::tlm + 2 h::tlm \Leftrightarrow pq.h.h::tlm $\vspace*{0.2cm}\\
				$> pqh2::tlm + cytb6f::tml \Leftrightarrow pqh2.cytb6f::tml$\\
				$> pq\{n\}.h\{+\}.h\{+\}.cytb6f(bhc\{2-\})::tlm \\ \hspace*{.9cm} \Leftrightarrow pq\{n\}.h\{n\}.h\{n\}.cytb6f(bhc\{n\})::tlm$
			\end{itemize}
	\end{itemize}
\end{itemize}

\newpage

\mysection{Matching and Replacement}

Let $\alpha$ be an agent and $\Gamma_s,~\Gamma_l,~\Gamma_r$ be given solution, left-hand-side and right-hand-side of a rule, respectively.

\mydefinition{Let $l_1$ and $l_2$ be lists. We define difference between two lists $l~=~l_1~-~l_2$ as $[ x~|~x~\not\in~l_1~\cap~l_2 ]$.}

\mydefinition{
With assumption all rules of form $\Gamma_l\Rightarrow\Gamma_r$ are well-formed, we can define \textbf{Matching} is relation $\models \subseteq \Gamma_s~\times~\Gamma_l $ defined via \textit{compatibility} operator. We denote $\Gamma_s$ \textit{matches} $\Gamma_l$ ($ \Gamma_s~\models~\Gamma_l$) iff $\forall \alpha_l \in \Gamma_l: \exists \alpha_s \in \Gamma_s: \alpha_s \lhd \alpha_l$.}


\textbf{Replacement} is a function $\Gamma_s~\times~\Gamma_r~\rightarrow~\Gamma_s[\Gamma_r]$. Each of the $\Gamma$ constructs might be one of atomic agent $a$, structure agent $\mathtt{T}$, complex agent $\mathtt{X}$, a list of them or $\emptyset$ (what follows from flattening of the rules). Before replacement can be applied, matching in left column must be satisfied. Then, five match/replacement categories can be distinguished:

\begin{enumerate}
\item \textit{state change}:

Let $\gamma^{l-r}$ is the difference between $\gamma^{l}$ and $\gamma^{r}$ compositions. Then $\gamma^{s^-}$ is part of $\gamma^{s}$ composition such that $\forall~\alpha_1~\in~\gamma^{s^-}~\exists~\alpha_2~\in~\gamma^{l-r}:~\alpha_1~\lhd~\alpha_2 $ and $\gamma^{s^+}$ is part of $\gamma^{s}$ composition such that $\forall~\alpha_1~\in~\gamma^{s^+}~\exists~\alpha_2~\in~\gamma^{s}:~\alpha_1~\lhd~\alpha_2 $ where $\gamma^{s^-}~\cup~\gamma^{s^+}~=~\gamma^{s}$ and $\gamma^{s^-}~\cap~\gamma^{s^+}~=~\emptyset$. 

\begin{itemize}
\item	\begin{tabular}{ l l }
				$ \mathtt{a}_s\{p\} \lhd \mathtt{a}_l $ & \hspace*{0.6 cm} $ \mathtt{a}_s\{p\}[\mathtt{a}_r\{t\}] = \mathtt{a}_s\{t\} $
		\end{tabular}
	  
where $ \mathtt{a}_s = \mathtt{a}_l = \mathtt{a}_r $.
\item   \begin{tabular}{ l l }
				$ \mathtt{T}_s \lhd \mathtt{T}_l $ & \hspace*{1 cm} $ \mathtt{T}_s[\mathtt{T}_r] = \mathtt{T}_s: \forall \mathtt{a}_r\{t\} \in \gamma^{l-r}: \mathtt{a}_s\{p\} \in \gamma^{s^-}: \mathtt{a}_s\{p\}[\mathtt{a}_r\{t\}] $
		\end{tabular}
		
where $\mathtt{T}_s = \tau_s(\gamma^{s})$, $\mathtt{T}_l = \tau_l(\gamma^{l})$, $\mathtt{T}_r = \tau_r(\gamma^{r})$ and $\tau_s = \tau_l = \tau_r$.

\item   \begin{tabular}{ l l }
			$ \mathtt{X}_s \lhd \mathtt{X}_l $ & \hspace*{1 cm} $ \mathtt{X}_s[\mathtt{X}_r] = \mathtt{X}_s: \forall \alpha_r \in \gamma^{l-r}: \alpha_s \in \gamma^{s^-} : \alpha_s[\alpha_r] $
		\end{tabular}
		
where $\mathtt{X}_s = \gamma^{s}$, $\mathtt{X}_l = \gamma^{l}$ and $\mathtt{X}_r = \gamma^{r}$.

\end{itemize}

\item \textit{complex formation}:
\begin{center}
\begin{tabular}{ l l }
	$ \Gamma_s \models \Gamma_l $ & \hspace*{1 cm} $\Gamma_s[\Gamma_r] = \gamma_f(\Gamma_s)$
\end{tabular}
\end{center}

\item \textit{complex dissociation}:
\begin{center}
\begin{tabular}{ l l }
	$  \Gamma_s \models \Gamma_l $ & \hspace*{1 cm} $ \gamma_f(\Gamma_s)[\Gamma_r] = \Gamma_s $
\end{tabular}
\end{center}

\item \textit{degradation}
\begin{center}
\begin{tabular}{ l l }
	$ \Gamma_s \models \Gamma_l $ & \hspace*{1 cm} $ \Gamma_s[\emptyset] = \emptyset $
\end{tabular}
\end{center}

\item \textit{translation}
\begin{center}
\begin{tabular}{ l l }
	$ \Gamma_s \models \Gamma_l ~(\emptyset \lhd \emptyset) $ & \hspace*{1 cm} $ \emptyset[\Gamma_r] = \Gamma_r $
\end{tabular}
\end{center}

\end{enumerate}

\textbf{TBA} lexicographically sorted is not working yet

\textbf{TBA} not sure about $\emptyset \lhd \emptyset$

\textbf{TBA} what to choose as $\Gamma_s$ ???

\end{document}